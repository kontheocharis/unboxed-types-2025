When the reduction rules of type theory become part of the static semantics, it
is impossible to always compute the memory layout of a type during compilation.
This happens when polymorphism is present but monomorphisation is not, and in
particular when dependent types are part of the mix, since the dependency might
be on runtime values.

In this ongoing work, we formulate a dependent type theory where sizes of types
are always known at compile-time, and thus compilation can target a language
like C. Indirection is opt-in and can be avoided, leading to efficient code that
enables cache locality. This is done by \emph{indexing} syntactic types by a
metatheoretic type describing memory layouts: $\Bytes$. This leads to a notion
of representation polymorphism. This has been explored before, but in the
context of non-dependent functional languages, and often comes with a heuristic
final pass which adds indirection to types whose size cannot be heuristically
determined. Our approach requires no heuristics, and extends to full dependent
types.

The staging view of \emph{two-level type theory} (2LTT) \cite{Annenkov2023-vk}
has been explored by Kov\'acs in the general setting \cite{Kovacs2022-rf} as
well as in the setting of closure-free functional programming
\cite{Kovacs2024-hn}. Inspired by a note in the afforementioned works, we embed
our unboxed type theory as the object language of a 2LTT, which allows us to
write type-safe metaprograms that compute representation-specific constructions.
For example, we can formulate a universe of flat protocol specifications in the
style of Allais \cite{Allais2023-zq}, and interpret it in the unboxed object
theory. We needn't compromise on the usage of dependent types either; as opposed
to \cite{Kovacs2024-hn}, our object theory is dependently typed and thus we can
encode higher-order polymorphic functions as part of the final program, but
without hiding data behind indirections.

\paragraph{Basic setup}
We formulate our system in an MLTT-style type theory, using the intrinsic QIIT
style of Altenkirch and Kaposi \cite{Altenkirch2016-zc}.
To do this, we first assume access to a metatheoretic type of
\(\Bytes\) with a signature
\[
0 : \Bytes \qquad 1 : \Bytes \qquad + : \Bytes \to \Bytes \to \Bytes \qquad \ptr : \Bytes \,.
\]
The constant \(\ptr\) defines the size of a pointer. Any model of
the signature above will suffice; such a model might encode a sophisticated
layout algorithm for example.

First, types are indexed not just by contexts, but also by
bytes:\footnote{For brevity we will not regard issues of universe sizing, but this can
be accomodated without issue.} $\Ty : \Con \to \Bytes \to \Set$.
We have the following basic type and term formers:
{\small%
\[
\begin{alignedat}{3}
&\univ_{\_} : \Bytes \to \Ty\ \Gamma\ 0 \quad  &&\El : \Tm\ \Gamma\ \univ_b \simeq \Ty\ \Gamma\ b : \code \quad  &&\langle-\rangle : \Tm\ \Gamma\ \univ_b \to \Tm\ \Gamma\ \univ \\
&\univ : \Ty\ \Gamma\ 0 \quad &&\El_\square : \Tm\ \Gamma\ \univ \to \Ty\ \Gamma\ \ptr\quad  && \syn{box} : \Tm\ \Gamma\ A \simeq \Tm\ \Gamma\ (\El_\square\ \langle\code\ A\rangle) : \syn{unbox} \\
\end{alignedat}
\]}

The type \(\univ\) is the type of codes for types of an unknown size, while the
\(\univ_b\) type is the type of codes for types of size \(b\). The \(\El\)
interpretation maps codes for types of size \(b\) to actual types of size \(b\),
while the \(\El_\square\) interpretation maps codes for any type to actual types
which \emph{box} their contents. In other words, internally, if \(A : \univ\),
then \(\El_\square\ A\) can be used as a type and at runtime the data of \(A\)
will be under a heap-allocated pointer indirection. On the other hand, if \(B :
\univ_b\), then \(\El\ B\) can be used as a type and at runtime the data of
\(B\) will be stored \emph{inline}, since it is known that \(B\) takes up
exactly \(b\) bytes. Codes for types of any kind take up no space at runtime
because they are erased.
Additionally, for any code \(t\) in \(\univ_b\), we can get a code
\(\langle t \rangle\) in \(\univ\) by `forgetting' that we know the size
of \(t\) is \(b\). Finally, we have boxing and unboxing operators for types of a known size.

\paragraph{Contexts and
substitutions}\label{contexts-and-substitutions}

Contexts $\Gamma$ store the size of each type, such that $|\Gamma| : \Bytes$ is the sum
of the sizes of its types: Extension is $\rhd\, : (\Gamma :
\Con) \to \{b : \Bytes\} \to \Ty\ \Gamma\ b \to \Con$. Substitutions are defined
as usual; we do not really need to augment the standard definition other than
adding implicit arguments for bytes. However, we must ensure that the action of
substitutions on types does not vary their sizes: $-[-] : \Ty\ \Gamma\ b \to
\Sub\ \Delta\ \Gamma \to \Ty\ \Delta\ b$.

\paragraph{$\Pi$ and $\Sigma$}\label{function-types}

This setup can be augmented with $\Pi$ and $\Sigma$ types, where the dependency
is \emph{uniform} with respect to layout:
\[
\Pi : (A : \Ty\ \Gamma\ a) \to \Ty\ (\Gamma \rhd A)\ b \to \Ty\ \Gamma\ \ptr \qquad
\Sigma : (A : \Ty\ \Gamma\ a) \to \Ty\ (\Gamma \rhd A)\ b \to \Ty\ \Gamma\ (a + b)
\]
A function's size in memory is just a pointer to its closure allocation which should
also contain the code pointer. For now we do not handle unboxed closure
captures, but the inputs and outputs are unboxed. Conversely, pairs are stored
inline; their size is the sum of the sizes of their components.

\paragraph{Padding type and unit}\label{padding-type-and-unit}

Sometimes we want to store some extra bytes to pad a smaller type in
order to reach a desired size. For this reason we add an extra type
former \[
\syn{Pad} : (b : \Bytes) \to \Ty\ \Gamma\ b \qquad \syn{pad} : \Tm\ \Gamma\ (\syn{Pad}\ b) \qquad
\syn{pad-$\eta$} : (t : \Tm\ \Gamma\ (\syn{Pad}\ b)) \to t = \syn{pad}
\] which is a generalisation of the unit type that takes up \(b\) bytes
in memory. We can then define \(\mathbb{1} := \syn{Pad}\ 0\) and
\(\syn{tt} := \syn{pad}\).

\paragraph{First-class layouts with staging}\label{layouts-staging}


\subsection{Example: Maybe as a tagged union}\label{maybe-as-a-tagged-union}

First, let's take a look at how to define the \(\syn{Maybe}\) type in
such a way that its data is stored contiguously as a tagged union. This
is similar to how languages such as Rust store it. \[
\begin{aligned}
&\syn{Maybe}_b : \Pi\ (T : \univ_b)\ \univ_{b + 1} \\
&\syn{Maybe}_b = \lambda\ T.\
\Sigma\ (x : \mathbb{2})\ (\syn{if}\ (x'.\ \univ_b)\ x\ T\ (\syn{Pad}\ b)) \\[1em]
&\syn{nothing}_b : \syn{Maybe}_b\ T \\
&\syn{nothing}_b = (\syn{false}, \syn{pad}) \\[1em]
&\syn{just}_b : T \to \syn{Maybe}_b\ T \\
&\syn{just}_b = \lambda\ t.\  (\syn{true}, t) \\[1em]
&\syn{maybe}_b : \{T : \univ_b\} \to (E : \syn{Maybe}_b\ T \to  \univ_c) \\
\quad & \to E\ \syn{nothing} \to ((t : T) \to E\ (\syn{just}\ t)) \\
\quad & \to (m : \syn{Maybe}_b\ T) \to E\ m \\
&\syn{maybe}_b = \lambda\ E\ n\ j\ m.\ \ldots
\end{aligned}
\]
