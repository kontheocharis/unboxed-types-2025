When the reduction rules of type theory become part of the static semantics, it
is impossible to always compute the memory layout of a type during compilation.
This happens when polymorphism is present but monomorphisation is not, and in
particular when dependent types are part of the mix, since the dependency might
be on runtime values.

In this ongoing work, we formulate a dependent type theory where sizes of types
are always known at compile-time, and thus compilation can target a language
like C. Indirection is opt-in and can be avoided, leading to efficient code that
enables cache locality. This is done by \emph{indexing} syntactic types by a
metatheoretic type describing memory layouts: $\Bytes$. This leads to a notion
of representation polymorphism. Unboxed data in functional languages has been
explored before \cite{Jones1991-ik,Downen2024-nk}, but results in overly restricted
polymorphism or complex theories with multiple levels of kinds separating values and computations, and without
support for full dependent types. Our approach is lightweight and extends to
full dependent types.

The staging view of \emph{two-level type theory} (2LTT) \cite{Annenkov2023-vk}
has been explored by Kov\'acs in the general setting \cite{Kovacs2022-rf} as
well as in the setting of closure-free functional programming
\cite{Kovacs2024-hn}. Inspired by a note in the afforementioned works, we can embed
our unboxed type theory as the object language of a 2LTT, which allows us to
write type-safe metaprograms that compute representation-specific constructions.
For example, we can formulate a universe of flat protocol specifications in the
style of Allais \cite{Allais2023-zq}, and interpret it in the unboxed object
theory.
We needn't compromise on the usage of dependent types either; as opposed to
\cite{Kovacs2024-hn}, the object theory is itself dependently typed and thus we
can encode unboxed higher-order polymorphic functions as part of the final
program, as long as all \emph{sizes} (but not necessarily all types) are known after staging.

\paragraph{Basic setup}
We formulate our system as an MLTT-style type theory, using the intrinsic QIIT
syntax of Altenkirch and Kaposi \cite{Altenkirch2016-zc}.
We assume access to a metatheoretic type of
\(\Bytes\) with
\[
0, 1 : \Bytes \qquad \qquad + : \Bytes \to \Bytes \to \Bytes \qquad \ptr : \Bytes \qquad \times : \syn{Nat} \to \Bytes \to \Bytes \,.
\]
The constant \(\ptr\) defines the size of a pointer. Any model of
the signature above will suffice; such a model might encode a sophisticated
layout algorithm with padding, for example.

First, types are indexed not just by contexts, but also by
bytes:\footnote{For brevity we will not regard issues of universe sizing, but this can
be accomodated without issue.} $\Ty : \Con \to \Bytes \to \Set$.
We have the following basic type and term formers:
{\small%
\[
\begin{alignedat}{3}
&\univ_{\_} : \Bytes \to \Ty\ \Gamma\ 0 \quad  &&\El : \Tm\ \Gamma\ \univ_b \simeq \Ty\ \Gamma\ b : \code \quad  &&\langle-\rangle : \Tm\ \Gamma\ \univ_b \to \Tm\ \Gamma\ \univ \\
&\univ : \Ty\ \Gamma\ 0 \quad &&\El_\square : \Tm\ \Gamma\ \univ \to \Ty\ \Gamma\ \ptr\quad  && \syn{box} : \Tm\ \Gamma\ A \simeq \Tm\ \Gamma\ (\El_\square\ \langle\code\ A\rangle) : \syn{unbox} \\
\end{alignedat}
\]}

The type \(\univ\) is the type of codes for types of an unknown size, while the
\(\univ_b\) type is the type of codes for types of size \(b\). The \(\El\)
interpretation maps codes for types of size \(b\) to actual types of size \(b\),
while the \(\El_\square\) interpretation maps codes for any type to actual types
which \emph{box} their contents. In other words, internally, if \(A : \univ\),
then \(\El_\square\ A\) can be used as a type and at runtime the data of \(A\)
will be under a heap-allocated pointer indirection. On the other hand, if \(B :
\univ_b\), then \(\El\ B\) can be used as a type and at runtime the data of
\(B\) will be stored \emph{inline}, since it is known that \(B\) takes up
exactly \(b\) bytes. Codes for types of any kind take up no space at runtime
because they are erased.
Additionally, for any code \(t\) in \(\univ_b\), we can get a code
\(\langle t \rangle\) in \(\univ\) by `forgetting' that the size
of \(t\) is \(b\). Finally, we have boxing and unboxing operators for types of a known size.
Contexts $\Gamma$ store the size of each type, such that $|\Gamma| : \Bytes$ is the sum
of the sizes of its types. The action of
substitutions on types does not vary their sizes: $-[-] : \Ty\ \Gamma\ b \to
\Sub\ \Delta\ \Gamma \to \Ty\ \Delta\ b$.

\paragraph{Unboxed pairs, boxed and unboxed closures}\label{function-types}

This setup can be augmented with $\Pi$ and $\Sigma$ types, where the dependency
is \emph{uniform} with respect to layout:
\begin{align*}
&\Pi_\square : (A : \Ty\ \Gamma\ a) \to \Ty\ (\Gamma \rhd A)\ b \to \Ty\ \Gamma\ (\ptr + \ptr) \\
&\Pi_k : (A : \Ty\ \Gamma\ a) \to \Ty\ (\Gamma \rhd A)\ b \to \Ty\ \Gamma\ (k + \ptr) \\
&\Sigma : (A : \Ty\ \Gamma\ a) \to \Ty\ (\Gamma \rhd A)\ b \to \Ty\ \Gamma\ (a + b)
\end{align*}
For functions, we have a choice of whether to box the captures or
not. In the latter case, we must annotate them with the size of their captures.
Conversely, pairs are stored inline; their size is the sum of the sizes of their
components. The term formers remain mostly unchanged; the only new case is
$\Pi_k$ whose lambda terms declare their captures through a substitution:
$\lambda : (\rho : \syn{Sub}\ \Gamma\ \Delta) \to \Tm\ (\Delta \rhd A)\ B \to \Tm\ \Gamma\ (\Pi_{|\Delta|}\ A\ B)[\rho]$.
In practice this substitution can be inferred from the surface program.

\paragraph{First-class byte values with staging}\label{layouts-staging}

This type theory can be embedded as an object language $\mathbb{O}$ of a 2LTT.
On the meta level, we have a type former $\mathbb{b} : \Ty_1\ \Gamma$ of byte
values, and term formers that mirror $\Bytes$.\footnote{%
More precisely, a free extension \cite{Yallop2018-zw} of $\Bytes$ by the
meta-level syntax.}
In an empty context, in a theory with canonicity, we get an
evaluation function $\syn{ev} : \Tm_1\ \cdot\ \mathbb{b} \to \Bytes$.
Adding $\Pi$ types to the meta language
allows abstraction over byte values. Moreover, the meta level has a type former
$\Uparrow_b : \Ty_0\ \Gamma\ b \to \Ty_1\ \Gamma$ for embedding $(b : \Tm\ \Gamma\ \mathbb{b})$-sized types from
the object fragment. If the final program is of the form $p :
\Tm_1\ \cdot\ (\Uparrow_k A)$, after staging we get an object term of size $\syn{ev}\ k$.

% \paragraph{Padding type and unit}\label{padding-type-and-unit}

% Sometimes we want to store some extra bytes to pad a smaller type in
% order to reach a desired size. For this reason we add an extra type
% former \[
% \syn{Pad} : (b : \Bytes) \to \Ty\ \Gamma\ b \qquad \syn{pad} : \Tm\ \Gamma\ (\syn{Pad}\ b) \qquad
% \syn{pad-$\eta$} : (t : \Tm\ \Gamma\ (\syn{Pad}\ b)) \to t = \syn{pad}
% \] which is a generalisation of the unit type that takes up \(b\) bytes
% in memory. We can then define \(\mathbb{1} := \syn{Pad}\ 0\) and
% \(\syn{tt} := \syn{pad}\).



\paragraph{Example: Maybe as a tagged union}\label{maybe-as-a-tagged-union}

Let's take a look at how to define the \(\syn{Maybe}\) type internally in
such a way that its data is stored contiguously as a tagged union without indirections.\footnote{This
is similar to the approach of languages such as Rust \cite{rust-option}.}.
We assume access to a type $\syn{Pad}\ b : \Ty\ \Gamma\ b$ which is the unit
type that takes up $b$ bytes, with sole constructor $\syn{pad}$ akin to
$\syn{tt}$.
\[
\begin{aligned}
&\syn{Maybe}_b : \Pi_0\ (T : \univ_b)\ \univ_{1 + b} \\
&\syn{Maybe}_b = \lambda\ T.\ \Sigma\ (x : \mathbb{2})\ (\syn{if}\ x\ \syn{then}\ T\ \syn{else}\ (\syn{Pad}\ b)) \\[1em]
&\begin{alignedat}{2}
&\syn{nothing}_b : \Pi_0\ (T : \univ_b)\ (\syn{Maybe}_b\ T)  \qquad && \syn{just}_b : \Pi_0\ (T : \univ_b)\ \Pi_0\ T\ (\syn{Maybe}_b\ T) \\
&\syn{nothing}_b = \lambda\ \_ .\ (\syn{false}, \syn{pad}) && \syn{just}_b = \lambda\ \_\ t.\  (\syn{true}, t)
\end{alignedat}
\end{aligned}
\]

\paragraph{Example: Arrays}\label{arrays}

We can extend the language with flat arrays of a statically known size $n : \syn{Nat}$ by
$\syn{Array}\ n : \Ty\ \Gamma\ b \to \Ty\ \Gamma\ (n \times b)$, as well
as of a runtime size $r : \Tm\ \Gamma\ \mathbb{N}$ by $\syn{DynArray}\ r : \Ty\ \Gamma\ b \to \Tm\ \Gamma\ \univ$
whose inhabitants can only be accessed under a box. For example,
\[
\Sigma\ (n : \mathbb{N})\ (\El_\square\ (\syn{DynArray}\ n\ (\syn{Array}\ 2\ (\Pi_k\ A\ B))))
\]
is the type of dynamically-sized arrays of pairs of closures from $A$ to $B$ of capture size $k$.
