\[
\newcommand{\syn}{\textsf}
\newcommand{\Ty}{\syn{Ty}}
\newcommand{\Tm}{\syn{Tm}}
\newcommand{\Con}{\syn{Con}}
\newcommand{\Tms}{\syn{Tms}}
\newcommand{\Tel}{\syn{Tel}}
\newcommand{\Sub}{\syn{Sub}}
\newcommand{\El}{\syn{El}}
\newcommand{\n}{\syn{n}}
\newcommand{\code}{\syn{code}}
\newcommand{\ap}{\syn{ap}}
\newcommand{\univ}{\mathcal{U}}
\newcommand{\ptr}{\syn{ptr}}
\newcommand{\Bytes}{\syn{Bytes}}
\newcommand{\meta}{\textbf}
\newcommand{\remb}{\urcorner}
\newcommand{\Set}{\meta{Set}}
\]

We formulate an MLTT-style type theory augmented to have unboxed
semantics. To do this, we first assume access to a metatheoretic type of
\(\Bytes\). This comes with a signature \[
0 : \Bytes \qquad 1 : \Bytes \qquad + : \Bytes \to \Bytes \to \Bytes \qquad \ptr : \Bytes \,,
\] where the \(0\) element should be neutral with respect to addition.
The type of natural numbers is a perfectly valid definition for this
type, but we abstract away from that for flexibility (perhaps we might
want the extended natural numbers for example). The constant \(\ptr\)
defines the size of a pointer. This will usually be 8 but might be
different depending on the target architecture. Generally we want to
make as few assumptions about the target as possible.

First, our types will be indexed not just by contexts, but also by
bytes: \[
\Ty : \Con \to \Bytes \to \Set \,.
\] In our type theory, we will not regard issues of universe sizing, as
this is orthogonal to our goals. We have the following basic type
formers: \[
\begin{aligned}
&\univ : \Ty\ \Gamma\ 0 \\
&\univ_{\_} : (b : \Bytes) \to \Ty\ \Gamma\ 0 \\
&\El : \Tm\ \Gamma\ \univ_b \to \Ty\ \Gamma\ b \\
&\El_\square : \Tm\ \Gamma\ \univ \to \Ty\ \Gamma\ \ptr \\
\end{aligned}
\] The type \(\univ\) is the type of codes for types that are not
necessarily sized, while the \(\univ_b\) type is the type of codes for
types of size \(b\). The \(\El\) interpretation maps codes for types of
size \(b\) to actual types of size \(b\), while the \(\El_\square\)
interpretation maps codes for any type to actual types which \emph{box}
their contents. In other words, internally, if \(A : \univ\), then
\(\El_\square\ A\) can be used as a type and at runtime the data of
\(A\) will be under a heap-allocated pointer indirection. On the other
hand, if \(B : \univ_b\), then \(\El\ B\) can be used as a type and at
runtime the data of \(B\) will be stored \emph{inline}, since it is
known that \(B\) takes up exactly \(b\) bytes. Codes for types of any
kind take up no space at runtime because there is no way to pattern
match on them.

On the term side, we have some formers for codes of types: \[
\begin{aligned}
\code : \Ty\ \Gamma\ b \to \Tm\ \Gamma\ \univ_b \\
\langle\_\rangle : \Tm\ \Gamma\ \univ_b \to \Tm\ \Gamma\ \univ
\end{aligned}
\] First, for any type \(A\) of size \(b\) we have \(\code\ A\) for the
code corresponding to \(A\), which itself is a term of type \(\univ_b\).
Additionally, for any code \(t\) in \(\univ_b\), we can get a code
\(\langle t \rangle\) in \(\univ\) by `forgetting' that we know the size
of \(t\) is \(b\). The sized term formers satisfy the definitional
equalities \[
\code\ (\El\ t) = t \qquad \El\ (\code\ A) = A \,.
\] Notice that we do not have a \(\code\) operator for types with
unknown size. This is because all types have a known size (types are
indexed by their size, after all!). With these formers, we can construct
a `boxing' type constructor as \[
\begin{aligned}
&\square : \Ty\ \Gamma\ b \to \Ty\ \Gamma\ \ptr \\
&\square\ A = \El_\square\ \langle \code\ A\rangle \,.
\end{aligned}
\] We can use this type constructor to box types of a \emph{known} size,
for example to improve performance of copying. We can also define boxing
and unboxing operations as \[
\syn{box} : \Tm\ \Gamma\ A \simeq \Tm\ \Gamma\ (\square\ A) : \syn{unbox} \\
\] related by a definitional isomorphism, as part of the syntax.

\subsection{Contexts and
substitutions}\label{contexts-and-substitutions}

We now augment our syntax with dependent function and pair types.
However, before we do that, we must be more explicit about contexts and
substitutions. First, we define contexts as \[
\Con : \Set \qquad \cdot : \Con \qquad
\rhd\, : (\Gamma : \Con) \to \{b : \Bytes\} \to \Ty\ \Gamma\ b \to \Con
\] Each member of a context has a well-defined size. As a result, we can
define the obvious projection function \(|\_|\) that takes a context to
its size (the sum of the sizes of its types). The fibres of this
projection are contexts indexed by size \(b\), which we denote
\(\Con\ b\).

Substitutions are defined as usual; we do not really need to augment the
standard definition. However, we must require that the action of
substitutions on terms and types does not vary their sizes. In other
words, \[
\_[\_] : \Ty\ \Gamma\ b \to \Sub\ \Delta\ \Gamma \to \Ty\ \Delta\ b \qquad
\_[\_] : \Tm\ \Gamma\ A \to (\sigma : \Sub\ \Delta\ \Gamma) \to \Tm\ \Delta\ A[\sigma] \,.
\] We also require that the type formers in the previous section, and
their attached sizes, are stable under substitutions.

\subsection{Function types}\label{function-types}

We define dependent function types with former \[
\Pi : (A : \Ty\ \Gamma\ a) \to \Ty\ (\Gamma \rhd A)\ b \to \Ty\ \Gamma\ \ptr
\] A function's size in memory is just a pointer; this pointer is a
pointer to a heap allocation that stores the captures of the functions
(under another pointer), and the code of the function itself. In this
system, we do not reason about closures; we expect that the runtime
handles that. One might be able to extend the type theory to reason
about sizes of captures, but our attempts resulted in complicated
theories. Nevertheless, we are able to model functions whose inputs and
outputs are \emph{unboxed}, which has the standard stack semantics of a
language like C. It is important that sizes are unaffected by
substitution, because then under a global context
\(\sigma : \Sub\ \cdot\ \Gamma\) we have that types in \(\Gamma\) of
size \(b\) indeed have size \(b\) at evaluation time. The term formers
for function types remain unchanged as usual: \[
\lambda : \Tm\ (\Gamma \rhd A)\ B \simeq \Tm\ \Gamma\ (\Pi\ A\ B) : \ap \,.
\]

\subsection{Pair types}\label{pair-types}

As opposed to function types, we are able to store pairs truly
contiguously if both sides have a known size. In other words, the type
former for pairs is \[
\Sigma : (A : \Ty\ \Gamma\ a) \to \Ty\ (\Gamma \rhd A)\ b \to \Ty\ \Gamma\ (a + b)
\] One might raise some objections about alignment in memory; surely
sometimes we want to align the fields such that in an iterated
\(\Sigma\) type all elements are equal distance away. However, since we
are working with an abstract \(\Bytes\) type we are free to define
addition as we please, and we do not require any properties other
neutrality of 0---it need not even be commutative! Therefore, we can
encode a sophisticated alignment algorithm as the addition \(+\)
operation of \(\Bytes\).

The introduction and elimination rules for pairs remain unchanged: \[
(\_,\_) : (t : \Tm\ \Gamma\ A) \times \Tm\ \Gamma\ B[a]
\simeq \Tm\ \Gamma\ (\Sigma\ A\ B) : \syn{pr}
\]

\subsection{Padding type and unit}\label{padding-type-and-unit}

Sometimes we want to store some extra bytes to pad a smaller type in
order to reach a desired size. For this reason we add an extra type
former \[
\syn{Pad} : (b : \Bytes) \to \Ty\ \Gamma\ b \qquad \syn{pad} : \Tm\ \Gamma\ (\syn{Pad}\ b) \qquad
\syn{pad-$\eta$} : (t : \Tm\ \Gamma\ (\syn{Pad}\ b)) \to t = \syn{pad}
\] which is a generalisation of the unit type that takes up \(b\) bytes
in memory. We can then define \(\mathbb{1} := \syn{Pad}\ 0\) and
\(\syn{tt} := \syn{pad}\).

\subsection{Examples}\label{examples}

We assume that definitions are able to abstract over finitely many byte
values, which we write as a subscript like \(\syn{Maybe}_b\). In
practice this is a straightforward extension of the syntax, similarly to
simple (non-first-class) universe polymorphism, that we leave as an
implementation detail. Each usage of a definition is instantiated to the
correct byte values through constraint solving. Since bytes are not
terms in the language but rather literals, we do not need to worry about
abstracting over bytes or other complicated features that might lead to
unknown type sizes at compile-time.

\subsubsection{Maybe as a tagged union}\label{maybe-as-a-tagged-union}

First, let's take a look at how to define the \(\syn{Maybe}\) type in
such a way that its data is stored contiguously as a tagged union. This
is similar to how languages such as Rust store it. \[
\begin{aligned}
&\syn{Maybe}_b : \Pi\ (T : \univ_b)\ \univ_{b + 1} \\
&\syn{Maybe}_b = \lambda\ T.\
\Sigma\ (x : \mathbb{2})\ (\syn{if}\ (x'.\ \univ_b)\ x\ T\ (\syn{Pad}\ b)) \\[1em]
&\syn{nothing}_b : \syn{Maybe}_b\ T \\
&\syn{nothing}_b = (\syn{false}, \syn{pad}) \\[1em]
&\syn{just}_b : T \to \syn{Maybe}_b\ T \\
&\syn{just}_b = \lambda\ t.\  (\syn{true}, t) \\[1em]
&\syn{maybe}_b : \{T : \univ_b\} \to (E : \syn{Maybe}_b\ T \to  \univ_c) \\
\quad & \to E\ \syn{nothing} \to ((t : T) \to E\ (\syn{just}\ t)) \\
\quad & \to (m : \syn{Maybe}_b\ T) \to E\ m \\
&\syn{maybe}_b = \lambda\ E\ n\ j\ m.\ \ldots
\end{aligned}
\] \#\#\# Byte buffers

\subsection{First-class byte sizes with
staging}\label{first-class-byte-sizes-with-staging}

If we embed the type theory defined above in another system, to form a
\emph{two-level} type theory, we can reason about byte values as part of
the language itself.

Consider another type theory whose types \(\syn{BTy}\) look like \[
\mathbb{b} : \syn{BTy}\ \Gamma \qquad
\Pi\mathbb{b} : (A : \syn{BTy}\ \Gamma) \to \syn{BTy}\ (\Gamma \rhd A) \to \syn{BTy}\ \Gamma
\] and whose terms abstract over values of \(\mathbb{b}\) only. This is
a simple second-order theory over an abstract type of bytes. The terms
look like \[
\_\urcorner : \syn{BTm}\ \Gamma\ \mathbb{b} \qquad
\Pi\mathbb{b} : (A : \syn{BTy}\ \Gamma) \to \syn{BTy}\ (\Gamma \rhd A) \to \syn{BTy}\ \Gamma
\]

\subsection{Implementation notes}\label{implementation-notes}

As usual we can omit codes and interpretations.

\subsection{Extensions}\label{extensions}

\begin{itemize}
\tightlist
\item
  First class abstraction over bytes.
\end{itemize}
